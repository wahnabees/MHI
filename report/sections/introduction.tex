\section{Introduction}

Human activity recognition (HAR) plays an important role in applications such
as surveillance and security, human–computer interaction, consumer devices,
sports analytics, automotive systems, and healthcare monitoring. A core
challenge in HAR is to represent the dynamics of human motion in a form that is
both compact and discriminative, while remaining robust to variations in
subjects, appearance, and viewpoint. Although deep learning methods have
become dominant in recent years, classical motion-template approaches continue
to offer interpretable and computationally efficient alternatives for
video-based action analysis.

Motion History Images (MHI) and Motion Energy Images (MEI), introduced in the
seminal work of Bobick and Davis~\cite{bobick2001}, convert a sequence of
frames into static temporal templates that summarize where motion has occurred
and how it evolves over time. These templates have been widely used in early
HAR pipelines due to their simplicity and their ability to encode coarse action
structure in a compact form.

Once constructed, MHIs and MEIs allow the extraction of global shape-based
features. Hu moments provide a lightweight, invariant descriptor that has been
used in template-matching and gesture-recognition tasks within classical
computer-vision systems \cite{ming2009hu}. More expressive local descriptors,
such as Histograms of Oriented Gradients (HOG)~\cite{dalal2005}, capture
fine-grained spatial information and were foundational to many pre-deep-learning
action-recognition methods. Although HOG is not the focus of this work, it
serves as a useful point of comparison for understanding the limitations of Hu
moments and for motivating future improvements.

This paper evaluates an MHI/MEI-based action-recognition system using three
classical classifiers: Support Vector Machines (SVM)~\cite{cortes1995support},
$k$-Nearest Neighbors (KNN), and a lightweight Multi-Layer Perceptron (MLP).
Each classifier is trained on Hu-moment descriptors extracted from motion
templates, enabling a controlled comparison of classifier behavior under a
low-dimensional, globally defined representation.

The system pipeline includes preprocessing, motion-template construction,
feature extraction, classifier training, and evaluation. The experimental
results highlight both the strengths and limitations of Hu-moment-based
representations, echoing earlier findings that silhouette-driven templates
capture coarse motion patterns but struggle with subtle differences among
similar actions~\cite{bobick2001}. These observations, together with trends in
modern spatiotemporal feature learning \cite{klaser2008}, point
toward promising future directions such as HOG-based motion descriptors,
trajectory-aligned features, or learned CNN representations, as well as
temporal modeling using probabilistic or recurrent architectures.
