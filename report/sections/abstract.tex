\begin{abstract}
    This paper presents a human activity recognition system that leverages Motion
    History Images (MHI) and Motion Energy Images (MEI) in combination with
    classical machine learning classifiers, including Support Vector Machines
    (SVM), $k$-Nearest Neighbors (KNN), and Multi-Layer Perceptrons (MLP). Using Hu
    moment descriptors extracted from the motion templates, we evaluate how each
    classifier performs when provided with low-dimensional, global shape features.
    We also examine the impact of parameter tuning—particularly motion thresholds
    and temporal decay—on the quality of the templates and the resulting changes in
    classification accuracy.
    
    Our experiments show that although all three classifiers can make use of Hu
    moment representations, their performance varies depending on their sensitivity
    to feature distribution and noise. These findings highlight both the strengths and limitations of
    compact temporal templates and motivate the exploration of richer feature
    representations, such as HOG or learned descriptors, as promising directions
    for future work.
\end{abstract}
    