\section{Improvements}
\label{sec:improvements}

After evaluating SVM, KNN, and MLP on Hu moment features, SVM consistently
offered the most stable performance. It was therefore selected as the primary
candidate for improvement. The refinements described below build on a baseline
Hu-only model that had already been tuned to a reasonable level before
additional enhancements were introduced.

\subsection{Feature Expansion: From 14D Hu Moments to a 20D Motion Descriptor}

The original feature representation consisted of 14 values: the seven Hu
moments extracted from the MHI and the seven Hu moments extracted from the MEI.
While compact and effective for capturing global shape, this 14D descriptor
omits several important characteristics of human motion, such as intensity,
localization, and temporal variability. To address this limitation, six
additional motion descriptors were introduced, expanding the feature vector
from 14 to 20 dimensions.

This expansion strengthens the representation in three key ways:

\begin{itemize}
    \item \textbf{Capturing motion magnitude:} Aggregated motion-pixel counts
    provide a direct measure of how much motion occurs in a frame, helping
    distinguish high-intensity actions (running, boxing) from lower-intensity
    ones (walking).

    \item \textbf{Encoding temporal dynamics:} Features such as the frame-to-frame
    change in the MHI and the short-term standard deviation of motion quantify
    how stable, periodic, or bursty an action is—properties not encoded by Hu
    moments.

    \item \textbf{Preserving spatial structure:} Lower-body motion, central-region
    motion, and central-to-side ratios describe how motion is distributed
    across the frame. These cues differentiate actions with similar overall
    silhouettes but different spatial emphasis (e.g., running vs.~boxing).
\end{itemize}

By augmenting the Hu feature space with these lightweight descriptors, the SVM
operates in a richer but still low-dimensional space where classes are more
separable. This expansion from 14D to 20D significantly enhances the model’s
ability to discriminate between actions with subtle motion differences, as
validated by improved validation and video-level accuracy.


\subsection{Parameter Calibration}

Several parameters controlling MHI/MEI construction were adjusted:

\begin{itemize}
    \item $\tau$ (decay): balanced temporal smoothing vs.~motion persistence,
    \item $\theta$ (threshold): reduced noise in motion masks,
    \item $k\_size$ (Gaussian blur): stabilized differencing,
    \item reset frequency: prevented MHI saturation.
\end{itemize}

These refinements produced cleaner templates and more reliable features, helping
the SVM separate visually similar actions such as walking, jogging, and
running.

\subsection{Performance Comparison}

Table~\ref{tab:improvement-comparison} summarizes the improvement over the
baseline Hu-only SVM. The largest gain appears in video-level accuracy, rising
from approximately 0.63 to over 0.80.

\begin{table}[t]
\centering
\caption{Comparison of SVM performance before and after adding auxiliary motion features.}
\label{tab:improvement-comparison}
\begin{tabular}{lcccc}
\toprule
\textbf{Model} & \textbf{Train} & \textbf{Val} & \textbf{Test} & \textbf{Video Acc.} \\
\midrule
Baseline SVM (Hu-only) & 0.764 & 0.577 & 0.514 & 0.627 \\
Improved SVM (Hu + extra feats) & 0.939 & 0.682 & 0.622 & 0.807 \\
\bottomrule
\end{tabular}
\end{table}

\subsection{Confusion Matrix Evidence}

Figure~\ref{fig:cm-improvements-svm} shows the confusion matrix of the improved
model. Compared to earlier runs, the diagonal is noticeably stronger and
confusions between locomotion classes are substantially reduced. This reflects
the added discriminative value of the auxiliary motion descriptors.

\begin{figure}[t]
    \centering    
    \includegraphics[width=\linewidth]{figures/metrics/svm-improved-test.png}
    \caption{Confusion matrix of the improved SVM classifier using Hu moments
    plus auxiliary motion features. Strengthened diagonal entries reflect
    clearer class separation.}
    \label{fig:cm-improvements-svm}
\end{figure}

\subsection{Summary}

Overall, the improvements result from:
\begin{itemize}
    \item a richer feature set augmenting Hu moments,
    \item calibrated motion-template parameters,
    \item cleaner and more stable MHI/MEI representations.
\end{itemize}

These changes enabled the SVM to outperform all earlier Hu-based models while
preserving interpretability and computational simplicity.
