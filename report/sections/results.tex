\section{Results}
\label{sec:results}

This section presents the quantitative outcomes obtained from the three
classifier configurations evaluated in this work. To avoid redundancy, we
summarize all measured performance metrics in a single consolidated table
(Table~\ref{tab:results-summary}). These values include training, validation,
test, and video-level accuracy for each classifier trained on Hu moment
descriptors extracted from MHI and MEI templates.

\begin{table}[t]
    \centering
    \caption{Performance results for all classifiers using Hu moment features.}
    \label{tab:results-summary}
    \begin{tabular}{lcccc}
    \toprule
    \textbf{Classifier} & \textbf{Train} & \textbf{Val} & \textbf{Test} & \textbf{Video Acc.} \\
    \midrule
    SVM (Hu)  & 0.764 & 0.577 & 0.514 & 0.627 \\
    KNN (Hu)  & 1.000 & 0.476 & 0.453 & 0.651 \\
    MLP (Hu)  & 0.748 & 0.553 & 0.512 & 0.627 \\
    \bottomrule
    \end{tabular}
\end{table}
    

For completeness, confusion matrices for each classifier configuration were
generated and are included as figures in the results section. These
matrices illustrate the distribution of predicted labels across classes but are
not interpreted here, as further discussion is provided in the Analysis section.

In addition to accuracy values, the trained pipelines for each experiment
(SVM, KNN, and MLP) were saved for reproducibility. The corresponding feature
matrix dimensions reflect the use of Hu moment descriptors, resulting in a
14-dimensional feature vector for every video frame.
