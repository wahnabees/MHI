\section{Analysis}
\label{sec:analysis}

To compare the behavior of the three classifiers, we consolidate the main
performance metrics in Table~\ref{tab:results-summary}. This provides a unified
view of training, validation, test, and video-level accuracy across SVM, KNN,
and MLP when all three models are trained using Hu moments extracted from MHI
and MEI. The table is reported in the Results section; here we focus on
interpreting the trends it reveals.

\subsection{Key Observations}

\paragraph*{Limited discriminative power of Hu features.}
Hu moments provide compact global shape descriptors, but they offer limited
ability to separate visually similar activities, a limitation noted in prior
template-based action recognition research~\cite{bobick2001}. Because all
classifiers rely on the same low-dimensional representation, their performance
is inherently constrained by the descriptive power of Hu moments.

\subsection{Effect of Subject and Scene Variability}

The dataset includes substantial intra-class variability: different subjects
perform the same action with variations in body shape, clothing, motion style,
and execution speed. Similar challenges in human action recognition datasets
have been documented in earlier evaluations~\cite{schuldt2004recognizing}.
Because Hu moment features extracted from MHI/MEI templates encode only coarse
global motion shape, they are not expressive enough to disentangle these
variations. Consequently, actions such as walking, jogging, and running—already
visually similar—become even harder to discriminate when performed by different
individuals under different conditions.

\paragraph*{KNN exhibits pronounced overfitting.}
KNN achieves nearly perfect training accuracy but shows a substantial drop in
validation and test performance. This behavior is consistent with classical
observations that KNN is highly sensitive to small variations when classes are
not well separated in feature space~\cite{cover1967nearest}. In our case,
overlapping Hu moment descriptors cause unstable nearest-neighbor
relationships, limiting generalization.

\paragraph*{SVM shows more stable generalization.}
SVM avoids the extreme overfitting observed in KNN and produces more balanced
performance across training and test sets. Prior work has shown that SVMs
generalize effectively in low-dimensional settings with overlapping features by
maximizing the decision margin~\cite{cortes1995support}. Nevertheless, their
accuracy remains constrained by the limited separability inherent in Hu-based
representations.

\paragraph*{MLP performs comparably but is feature-limited.}
Although MLPs can model nonlinear relationships, their performance does not
substantially exceed that of SVM when trained on simple descriptors. Similar
effects have been reported in earlier neural-network–based action analysis,
where limited features restrict the benefits of nonlinear models. 
Because the MLP receives only 14 global shape features, its expressive capacity
is underutilized.

\paragraph*{Upper-bound limitations.}
Across all three classifiers, accuracy remains below higher performance
thresholds due to factors commonly cited in template-based action-recognition
systems: (i) compression of temporal information in silhouette templates,
(ii) similarity among certain actions (walking, jogging, running), and
(iii) subject-level variability~\cite{bobick2001, schuldt2004recognizing}. These
factors contribute to classification ambiguity that is not fully resolved by
Hu-based descriptors.

Taken together, the analysis highlights that while classifier choice influences
generalization behavior, the primary performance limitations stem from the
restricted descriptive capacity of Hu moments.

\subsection{Confusion Matrix Visualization}

To further illustrate how each classifier distributes predictions across
activity classes, Figures~\ref{fig:cm-svm}, \ref{fig:cm-knn}, and
\ref{fig:cm-mlp} present the confusion matrices for SVM, KNN, and MLP. These
matrices visualize class-specific behavior, including which actions are
consistently recognized and where misclassifications occur.


\begin{figure}[t]
    \centering
    \includegraphics[width=\linewidth]{figures/metrics/cm_test_svm.png}
    \caption{SVM classifier using Hu moment features.}
    \label{fig:cm-svm}
\end{figure}

\begin{figure}[t]
    \centering
    \includegraphics[width=\linewidth]{figures/metrics/cm_test_knn.png}
    \caption{KNN classifier using Hu moment features.}
    \label{fig:cm-knn}
\end{figure}

\begin{figure}[t]
    \centering
    \includegraphics[width=\linewidth]{figures/metrics/cm_test_mlp.png}
    \caption{MLP classifier using Hu moment features.}
    \label{fig:cm-mlp}
\end{figure}
