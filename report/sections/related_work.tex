\section{Related Work}
\label{sec:related}

\subsection{Motion Energy Images (MEI)}

Motion Energy Images (MEI)~\cite{bobick2001}, 
capture the spatial extent of motion over the most recent $\tau$ frames. MEI is 
defined as a binary motion template:

\begin{equation}
E_{\tau}(x,y,t) =
\begin{cases}
1, & \text{if motion occurs in } [t-\tau, t], \\
0, & \text{otherwise}.
\end{cases}
\end{equation}

In this formulation, motion is detected through thresholded frame differencing,
following the classical temporal-template approach \cite{bobick2001}:

\begin{equation}
D(x,y,t) =
\begin{cases}
1, & |I(x,y,t) - I(x,y,t-1)| > \theta, \\
0, & \text{otherwise}.
\end{cases}
\end{equation}


\subsubsection*{MEI Templates Across Six Actions}

Figure~\ref{fig:mei-six} shows MEI templates for six representative actions
(walking, jogging, running, boxing, hand waving, and hand clapping). These
binary silhouettes highlight the overall spatial footprint of motion accumulated
over time, revealing differences in limb usage, displacement, and body posture
across actions.For walking, jogging, and running, the MEIs appear as large white blobs because the actor moves left-to-right and back across the frame. Since MEI represents the union of all motion regions over time, any pixel touched during these actions becomes active, resulting in a broad motion band.

\begin{figure}[t]
    \centering

    \begin{subfigure}[b]{0.3\linewidth}
        \centering
        \includegraphics[width=\linewidth]{figures/evidences/walking_MEI.png}
        \caption{Walking}
    \end{subfigure}
    \begin{subfigure}[b]{0.3\linewidth}
        \centering
        \includegraphics[width=\linewidth]{figures/evidences/jogging_MEI.png}
        \caption{Jogging}
    \end{subfigure}
    \begin{subfigure}[b]{0.3\linewidth}
        \centering
        \includegraphics[width=\linewidth]{figures/evidences/running_MEI.png}
        \caption{Running}
    \end{subfigure}

    \vspace{2pt}

    \begin{subfigure}[b]{0.3\linewidth}
        \centering
        \includegraphics[width=\linewidth]{figures/evidences/boxing_MEI.png}
        \caption{Boxing}
    \end{subfigure}
    \begin{subfigure}[b]{0.3\linewidth}
        \centering
        \includegraphics[width=\linewidth]{figures/evidences/waving_MEI.png}
        \caption{Hand Waving}
    \end{subfigure}
    \begin{subfigure}[b]{0.3\linewidth}
        \centering
        \includegraphics[width=\linewidth]{figures/evidences/clapping_MEI.png}
        \caption{Hand Clapping}
    \end{subfigure}

    \caption{MEI templates for six actions}
    \label{fig:mei-six}
\end{figure}



\subsection{Motion History Images (MHI)}

Motion History Images (MHI) encode the recency of motion by assigning larger
intensity values to pixels where motion has occurred more recently:

\begin{equation}
M_{\tau}(x,y,t)=
\begin{cases}
\tau, & \text{if } B_{\tau}(x,y)=1, \\
\max\{0, M_{\tau}(x,y,t-1)-1,0\}, & \text{if } B_{\tau}(x,y)=0.
\end{cases}
\end{equation}

\subsubsection*{Binary Motion Mask and Background Subtraction}

Both MEI and MHI use a binary motion mask derived from frame differencing:

\begin{equation}
B_{\tau}(x,y,t) =
\begin{cases}
1, & |I{\tau}(x,y) - I_{\tau-1}(x,y)|\geq \theta, \\
0, & \text{otherwise}.
\end{cases}
\end{equation}

This mask identifies regions of movement by comparing consecutive frames.
Light smoothing can be applied before differencing to suppress noise.  
Where MEI simply accumulates these binary masks, MHI applies temporal decay,
producing a gradient that reflects *how motion evolves over time*.

\subsection{Preliminaries: Image Moments and Normalized Central Moments}

The Hu moment invariants used in this work are derived from standard image
moment definitions~\cite{hu1962}. These expressions describe how spatial
information in an image is aggregated and normalized to achieve invariance to
translation, rotation, and scale.

\paragraph*{Raw Image Moments.}
Given an image intensity function $I(x,y)$, the $(p,q)$-th raw moment is
defined as:
\begin{equation}
M_{pq} = \sum_{x}\sum_{y} x^{p}y^{q} I(x,y).
\end{equation}

\paragraph*{Centroid of the Image.}
The centroid $(\bar{x}, \bar{y})$ is computed from the first-order raw moments:
\begin{equation}
\bar{x} = \frac{M_{10}}{M_{00}}, \qquad
\bar{y} = \frac{M_{01}}{M_{00}}.
\end{equation}

\paragraph*{Central Moments.}
Translation-invariant central moments are computed by shifting coordinates
relative to the centroid:
\begin{equation}
\mu_{pq} = \sum_{x}\sum_{y} (x-\bar{x})^{p}(y-\bar{y})^{q} I(x,y).
\end{equation}

\paragraph*{Normalized Central Moments.}
Scale-invariant normalized central moments are obtained as:
\begin{equation}
\nu_{pq} = 
\frac{\mu_{pq}}
     {\mu_{00}^{\,1+\frac{p+q}{2}}}.
\end{equation}

These normalized moments form the basis for the Hu invariant descriptors used
in the next subsection, allowing activity templates such as MHI and MEI to be
represented using compact shape measures that are invariant to geometric
transformations.


\subsubsection*{MHI Templates Across Six Actions}

Figure~\ref{fig:mhi-six} displays MHI templates for the same six actions.
Unlike MEI—which only shows where movement occurred—MHI distinguishes between
recent and older motion. Brighter areas correspond to more recent movement,
making MHI a richer temporal descriptor.

\begin{figure}[t]
    \centering

    \begin{subfigure}[b]{0.3\linewidth}
        \centering
        \includegraphics[width=\linewidth]{figures/evidences/walking_MHI.png}
        \caption{Walking}
    \end{subfigure}
    \begin{subfigure}[b]{0.3\linewidth}
        \centering
        \includegraphics[width=\linewidth]{figures/evidences/jogging_MHI.png}
        \caption{Jogging}
    \end{subfigure}
    \begin{subfigure}[b]{0.3\linewidth}
        \centering
        \includegraphics[width=\linewidth]{figures/evidences/running_MHI.png}
        \caption{Running}
    \end{subfigure}

    \vspace{2pt}

    \begin{subfigure}[b]{0.3\linewidth}
        \centering
        \includegraphics[width=\linewidth]{figures/evidences/boxing_MHI.png}
        \caption{Boxing}
    \end{subfigure}
    \begin{subfigure}[b]{0.3\linewidth}
        \centering
        \includegraphics[width=\linewidth]{figures/evidences/waving_MHI.png}
        \caption{Hand Waving}
    \end{subfigure}
    \begin{subfigure}[b]{0.3\linewidth}
        \centering
        \includegraphics[width=\linewidth]{figures/evidences/clapping_MHI.png}
        \caption{Hand Clapping}
    \end{subfigure}

    \caption{MHI templates for six actions.}
    \label{fig:mhi-six}
\end{figure}



\subsection*{Evidence: Binary Motion Detection Across Time}

To demonstrate the foundation upon which MEI and MHI are constructed,
Figure~\ref{fig:binary-evidence} presents binary motion masks extracted at
three different time instants\cite{bobick2001}. Each image is produced by thresholding pixel
differences between consecutive frames, highlighting only the regions where
motion occurred at that moment.

These examples show how different actions produce distinct temporal patterns:
locomotion actions such as walking and jogging generate smoother, periodic
silhouettes, whereas high-frequency actions like boxing or hand clapping
produce denser, rapidly changing motion regions.

\begin{figure}[t]
    \centering

    % Disable (a), (b), (c) numbering
    \captionsetup[subfigure]{labelformat=empty}

    % Row 1
    \begin{subfigure}[b]{0.3\linewidth}
        \centering
        \includegraphics[width=\linewidth]{figures/evidences/walking.png}
        \caption{Walking (30,60,90)}
    \end{subfigure}
    \begin{subfigure}[b]{0.3\linewidth}
        \centering
        \includegraphics[width=\linewidth]{figures/evidences/jogging.png}
        \caption{Jogging (30,60,90)}
    \end{subfigure}
    \begin{subfigure}[b]{0.3\linewidth}
        \centering
        \includegraphics[width=\linewidth]{figures/evidences/running.png}
        \caption{Running (5,10,15)}
    \end{subfigure}

    \vspace{2pt}

    % Row 2
    \begin{subfigure}[b]{0.3\linewidth}
        \centering
        \includegraphics[width=\linewidth]{figures/evidences/boxing.png}
        \caption{Boxing (30,60,90)}
    \end{subfigure}
    \begin{subfigure}[b]{0.3\linewidth}
        \centering
        \includegraphics[width=\linewidth]{figures/evidences/waving.png}
        \caption{Waving (30,60,90)}
    \end{subfigure}
    \begin{subfigure}[b]{0.3\linewidth}
        \centering
        \includegraphics[width=\linewidth]{figures/evidences/clapping.png}
        \caption{Clapping(7,14,21)}
    \end{subfigure}

    {\small
    \caption{Binary motion evidence at different frames is indicated in braces for each action sequence.
    \label{fig:binary-evidence}
}}

\end{figure}





